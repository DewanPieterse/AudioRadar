\chapter{Recommendations}\label{chap:Recommendations}
Following the conclusion of the project, the following recommendations can be made where future projects can benefit and improve on the educational, embedded, audio radar.

\begin{enumerate}
\item Use a microphone with a greater sensitivity to quieter sounds without using automatic gain.
\item Improve on using the USB sound card by researching audio-specific, Raspberry Pi compatible, ADCs.
\item Improve the signal processing algorithms to improve speed when executing filters and FFTs.
\item Investigate the use of higher or lower frequencies in the audible spectrum for implementation to improve range and accuracy.
\item Investigate the usage of Notch, Band Pass, Low-Pass and High-Pass filters to improve their effectiveness in isolating movement in the Continuous Wave radar mode.
\item Develop printed circuit boards for all of the Veroboard circuits to improve the physical appearance and functionality of the radar.
\item Investigate other embedded platforms, like the Raspberry Pi 4B with up to 4GB RAM, to do the signal processing and improve overall speed.
\item Explore techniques used to develop a spectrogram function other than a build-in package's version to improve the resolution of the spectrogram.
\item Investigate how to suppress signals of power less than a certain value in different sound level conditions to improve the Range-Doppler Map.
\item Improve the webserver to display more interactive content with the specific mode's results that is showing on the results page.
\item Change the design to incorporate an Ethernet CAT5 cable to be internet-connected to the Raspberry Pi with the audio socket circuits fixed down. This would enable testing the radar while protecting the components.
\end{enumerate}


\newpage